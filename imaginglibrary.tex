\documentclass[a4paper]{article}

\usepackage[top=1in,bottom=1.25in,left=1.25in,right=1.25in]{geometry}

\hyphenation{op-tical net-works semi-conduc-tor IEEEtran}

\title{\LARGE \bf
Reference on\\Imaging Science and Engineering
}

\author{Fan Yang}

\begin{document}

\maketitle

\begin{abstract}
Imaging devices lie in the foremost phase in the pipeline of the artificial vision system that mimic human vision system. 
The quality of the acquisition and the in-place processing are at the stake of the computation that take place in the succeeding steps. 
This is a self-assembled reference indexing article in the field of imaging science and engineering, plus to some extent on the relative topics in the field of vision science, optics, psychophysics, and computer science. 
The collected literature are organized into different categories and are updated timely, I'll try my best to gurantee this. 
\end{abstract}

Imaging, especially color imaging, is an important and thriving field with the aim of acquiring high-quality images for both human and machines.

The literature are collected and organized into the following categories:
\begin{enumerate}
\item Scene modeling;
\item Human vision system;
\item Optical imaging;
\item Electronical imaging; and
\item Digital image processing.
\end{enumerate}
Each piece of the literature is listed with a few sentences (at least one) to introduce it. 

\section{Scene modeling}

Imaging devices capture digital/analogue images of the selected part of the natural/synthetic scene. 
The underlying principle is the natural law of reflectance of light. 
Light propagates in media such as air, water, and many other objects that are transparent. 
When travelling in the air, light may encounter different weather condition that caused the illumination complex. 
Also, in different time of a day on this planet, natural illuminant (i.e., sun in the day, and stars at night) fires different light. 

Radiometric, photometric, and colorimetric

Typical illuminants: color temperature,,,

Underwater: 

Bad weather conditions: haze, fog, rain

Misc: sky

\section{Human vision system}

\subsection{Lightness constancy}

\subsection{Color constancy}

\section{Optical imaging}

About lenses

\section{Electronical imaging}

\subsection{APS vs. DPS}

\subsection{Noise}

\section{Digital image processing}

\subsection{Color management}

\subsubsection{Tone reproduction}

Improved bilateral filtering by Chen {\it et al.} \cite{Chen:2007}. 

Real-time noise-aware tone mapping by Eilertsen {\it et al.} \cite{Eilertsen2015}, with supplementary material in \cite{Eilertsen2015supp}.

\subsubsection{Illumination estimation}

Survey paper written by Foster \cite{Foster2011}, Gijsenij {\it et al.} \cite{Gijsenij2011}, Barnard {\it et al.} \cite{Barnard2002a,Barnard2002b}.

Gray world assumption by Buchsbaum \cite{Buchsbaum1980}. 

\subsubsection{Chromatic adaptation}

\subsubsection{Color appearance models}

\subsubsection{Color appearance vs. Tone reproduction}

\subsection{Restoration \& Enhancement}

\subsubsection{Haze removal}

\subsubsection{Raindrop removal}

\subsubsection{Shadow removal}

\subsubsection{Noise reduction}

\subsubsection{Super resolution}

\section{Conclusions}

\bibliographystyle{acm}
\bibliography{imaginglibrary}

\end{document}